\documentclass[10pt,a4paper,colorlinks,linkcolor=cyan,urlcolor=cyan]{moderncv}

\usepackage{verbatim}

% moderncv themes
\moderncvtheme[green]{casual}

% character encoding
\usepackage[utf8]{inputenc}

% adjust the page margins
\usepackage[scale=0.9, top=1.5cm, bottom=1.5cm, includefoot]{geometry}
\recomputelengths

% preamble
\usepackage[stable]{footmisc}
\usepackage{datetime}
\usepackage{comment}
\usepackage{url}
\def\dates[#1.#2-#3.#4]{\yearabove{\shortmonthname[#1]}{#2}--\yearabove{\shortmonthname[#3]}{#4}}
\newcommand{\yearabove}[2]{\parbox[t]{10mm}{\centering{#2\par\vspace{-2mm} \scriptsize{#1}}}}
\renewcommand{\thefootnote}{*}

% personal data
\firstname{Hiroki}
\familyname{Tanabe}
\title{Data Scientist / PhD in Computer Science / YouTuber}
\IfFileExists{.personal_data.tex}{
	\input{.personal_data.tex}
}{
	\address{***DUMMY ADDRESS***}{}
\mobile{***-****-****}
\email{***dummy***@example.com}

}
\email{tanabe.hiroki.45n@kyoto-u.jp}
\social[linkedin]{hiroki-tanabe}
\social[twitter]{zalgo3}
\social[github]{zalgo3}

\begin{document}
\maketitle

\section{Experience}
\cventry{\yearabove{\shortmonthname[10]}{2022}--Present}{Data Scientist}{LY Corporation}{Osaka, Japan}{}{
	As a Data Scientist, I am involved in developing conversion rate (CVR) prediction models for our advertising platform, which handles hundreds of millions in daily sales. These models are essential for maintaining advertiser KPIs while maximizing budget utilization, indirectly contributing to revenue growth through optimized ad placements.\endgraf
	I contribute to the full lifecycle of these models, from data aggregation using PySpark, Trino, and Hive, to training with TensorFlow, XGBoost, and LightGBM, and finally to deployment on object storage for inference via TensorFlow Serving and Triton. This process is streamlined through an MLOps pipeline utilizing Airflow and Argo Workflows, ensuring regular updates and deployments.\endgraf
	Conducting A/B testing is another critical aspect of my role, allowing us to empirically assess the performance of our models on the platform. This testing ensures that only the most effective models are deployed, directly impacting our automated bidding service's efficiency and the overall success of ad campaigns.\endgraf
	Additionally, I have developed a dashboard using Streamlit to provide insights into model performance and operational metrics. This tool aids in the continuous monitoring and improvement of our models and workflows, enhancing decision-making processes.
}
\cventry{\yearabove{\shortmonthname[2]}{2017}--Present}{YouTuber}{}{}{}{
    I have \httplink[a YouTube channel]{youtube.com/@zalgo_video} with 19,000 subscribers, where I upload music and gaming videos.\endgraf
I also develop \httplink[an AI virtual YouTuber channel]{youtube.com/@MonomousuYT}. This channel uses a Python-based AI system to automatically generate aspects of the videos, including chat text, audio, motion, and subtitles. In particular, the OpenAI API is used to generate the conversation text. The AI system runs in an environment managed by the Azure Container Instances.
}
\cventry{\dates[04.2020-09.2022]}{Research Fellowship for Young Scientists (DC1)}{Japan Society for the Promotion of Science}{Tokyo, Japan}{}{
	Selected from the top 20\% of Ph.D. applicants, this fellowship allowed me to pursue a research topic in multiobjective optimization algorithms.\endgraf
	My research focused on the development, theoretical analysis, and implementation of first-order multiobjective optimization methods. In contrast to common approaches that optimize a weighted sum of objectives, the algorithms I developed handle the optimization without predefining the weighting parameters. This approach is particularly advantageous when dealing with non-convex objective functions or when simple weights for the objectives are difficult to assign.\endgraf
	I was the primary contributor to all aspects of the research, including problem definition, mathematical proofs, numerical experiments, and paper writing, working primarily for discussion with my advisors. My work resulted in the submission of five papers to peer-reviewed journals, four of which were accepted.\endgraf
	An essential part of my research was the translation of theory into practice. To this end, I implemented the proposed methods as a Python module, published it on GitHub and PyPI under the MIT license, and enabled others in the field to use the results of my research. This ensures not only the applicability of my results, but also their potential for further development and iteration in the broader research community.
}
\cventry{\dates[02.2019-03.2020]}{Data Scientist Internship}{HACARUS Inc.}{Kyoto, Japan}{}{
	I have designed and developed several machine learning models for the manufacturing and agricultural industries, mainly by analyzing a small number of image datasets, typically using Python. In particular, I developed models for surface defect detection using various techniques, including dictionary learning and fused lasso, and hyperspectral image segmentation using online dictionary learning and typical classification algorithms (e.g., support vector machine, random forest). Reviewed scientific papers and implemented algorithms from them using some machine learning libraries (e.g., scikit-learn), image processing libraries (e.g., open-cv, scikit-image), data visualization libraries (e.g., matplotlib, seaborn), and notebook environments (e.g., Jupyter).\endgraf
	\textbf{More minor tasks included}: Added some new algorithms (e.g., alternating direction methods of multipliers for the generalized lasso, matching pursuit for sparse coding) to the open source library ``\httplink[spm-image]{github.com/hacarus/spm-image}'', a scikit-learn compatible library for sparse modeling and compressive sensing. Implemented batch-OMP, an improved version of the orthogonal matching pursuit algorithm, to contribute to the research project of NEDO (the New Energy and Development Organization), Japan's largest government R\&D organization. I developed the front-end of an in-house web application that generates an interactive visualization of the quality of image datasets using flasks, pandas, and bokeh.
}

\section{Education~\footnotemark}
\cventry{\dates[10.2019-09.2022]}{PhD, Computer Science}{Kyoto University}{Kyoto, Japan}{}{}
\cventry{\dates[04.2018-09.2019]}{MSc, Computer Science}{Kyoto University}{Kyoto, Japan}{Valedictorian, early degree completion}{}
\cventry{\dates[04.2014-03.2018]}{BSc, Computer Science}{Kyoto University}{Kyoto, Japan}{Valedictorian}{}
\footnotetext{In Japan, Doctorate Degree = Doctor of Philosophy (PhD) = Doctor of Science (D.Sc.), Master's Degree = Master of Science (MSc), Bachelor's Degree = Bachelor of Science (BSc), and Informatics = Information Science = Computer Science are often used interchangeably. For this reason, PhD, MSc, BSc, and Computer Science are used here as universal translations to reduce confusion.}

\section{Skills Summary}
\subsection{Technical}
\cvcomputer{Industry Knowledge}{
	Optimization, Machine Learning, Data Engineering, DevOps, MLOps
}{Tools \& Technologies}{
	Python, SQL, Hive, Hadoop, Spark, Trino, Git, Docker, Kubernetes
}

\subsection{Language}
\cvcomputer{English}{Professional working proficiency}{Japanese}{Native or bilingual proficiency}

\section{Publications}
\subsection{Papers}
\cvlistitem{Hiroki Tanabe, Ellen H. Fukuda, and Nobuo Yamashita, New merit functions for multiobjective optimization and their properties. \textit{To appear in Optimization}, 2023. [\httplink[doi]{doi.org/10.1080/02331934.2023.2232794}, \httplink[pdf]{arxiv.org/abs/2010.09333}]}
\cvlistitem{Hiroki Tanabe, Ellen H. Fukuda, and Nobuo Yamashita, An accelerated proximal gradient method for multiobjective optimization. \textit{Computational Optimization and Applications}, 86, pp.~421-455, 2023. [\httplink[doi]{doi.org/10.1007/s10589-023-00497-w}, \httplink[pdf]{rdcu.be/debiq}]}
\cvlistitem{Hiroki Tanabe, Ellen H. Fukuda, and Nobuo Yamashita, Convergence rates analysis of a multiobjective proximal gradient method. \textit{Optimization Letters}, 17, pp.~333--350, 2023. [\httplink[doi]{doi.org/10.1007/s11590-022-01877-7}, \httplink[pdf]{rdcu.be/cL1Kg}]}
\cvlistitem{Hiroki Tanabe, Ellen H. Fukuda, and Nobuo Yamashita, Proximal gradient methods for multiobjective optimization and their applications, \textit{Computational Optimization and Applications}, 72(2), pp.~339--361, 2019. [\httplink[doi]{doi.org/10.1007/s10589-018-0043-x}, \httplink[pdf]{rdcu.be/bav7Q}]}
\cvlistitem{Hiroki Tanabe, Ellen H. Fukuda, and Nobuo Yamashita, A globally convergent fast iterative shrinkage-thresholding algorithm with a new momentum factor for single and multi-objective convex optimization. \textit{Submitted}, 2024. [\httplink[pdf]{arxiv.org/abs/2205.05262}]}

\subsection{Books}
\cvlistitem{Takashi Someda, Naoki Kitora, Ippei Usami, Ryuji Masui, and Hiroki Tanabe, An Introduction to Sparse Modeling for IT Engineers (in Japanese), Shoeisha, 2021.}

\section{References}
Available upon request.

\end{document}
